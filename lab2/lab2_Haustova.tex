\documentclass[a4paper, 14pt]{extarticle}

% Поля
%--------------------------------------
\usepackage{geometry}
\geometry{a4paper,tmargin=2cm,bmargin=2cm,lmargin=3cm,rmargin=1cm}
%--------------------------------------


%Russian-specific packages
%--------------------------------------
\usepackage[T2A]{fontenc}
\usepackage[utf8]{inputenc} 
\usepackage[english, main=russian]{babel}
%--------------------------------------

\usepackage{textcomp}

% Красная строка
%--------------------------------------
\usepackage{indentfirst}               
%--------------------------------------             


%Graphics
%--------------------------------------
\usepackage{graphicx}
\usepackage{float}
\graphicspath{ {./images/} }
\usepackage{wrapfig}
%--------------------------------------

% Полуторный интервал
%--------------------------------------
\linespread{1.3}                    
%--------------------------------------

%Выравнивание и переносы
%--------------------------------------
% Избавляемся от переполнений
\sloppy
% Запрещаем разрыв страницы после первой строки абзаца
\clubpenalty=10000
% Запрещаем разрыв страницы после последней строки абзаца
\widowpenalty=10000
%--------------------------------------

%Списки
\usepackage{enumitem}

%Подписи
\usepackage{caption} 

%Гиперссылки
\usepackage{hyperref}

\hypersetup {
	unicode=true
}

%Рисунки
%--------------------------------------
\DeclareCaptionLabelSeparator*{emdash}{~--- }
\captionsetup[figure]{labelsep=emdash,font=onehalfspacing,position=bottom}
%--------------------------------------

\usepackage{tempora}
\usepackage{amsmath}
\usepackage{color}
\usepackage{listings}
\lstset{
  belowcaptionskip=1\baselineskip,
  breaklines=true,
  frame=L,
  xleftmargin=\parindent,
  language=Python,
  showstringspaces=false,
  basicstyle=\footnotesize\ttfamily,
  keywordstyle=\bfseries\color{blue},
  commentstyle=\itshape\color{purple},
  identifierstyle=\color{black},
  stringstyle=\color{red},
}

%--------------------------------------
%			НАЧАЛО ДОКУМЕНТА
%--------------------------------------

\begin{document}

%--------------------------------------
%			ТИТУЛЬНЫЙ ЛИСТ
%--------------------------------------
\begin{titlepage}
\thispagestyle{empty}
\newpage


%Шапка титульного листа
%--------------------------------------
\vspace*{-60pt}
\hspace{-65pt}
\begin{minipage}{0.3\textwidth}
\hspace*{-20pt}\centering
\includegraphics[width=\textwidth]{emblem}
\end{minipage}
\begin{minipage}{0.67\textwidth}\small \textbf{
\vspace*{-0.7ex}
\hspace*{-6pt}\centerline{Министерство науки и высшего образования Российской Федерации}
\vspace*{-0.7ex}
\centerline{Федеральное государственное бюджетное образовательное учреждение }
\vspace*{-0.7ex}
\centerline{высшего образования}
\vspace*{-0.7ex}
\centerline{<<Московский государственный технический университет}
\vspace*{-0.7ex}
\centerline{имени .. Баумана}
\vspace*{-0.7ex}
\centerline{(национальный исследовательский университет)>>}
\vspace*{-0.7ex}
\centerline{(МГТУ им. Н.Э. Баумана)}}
\end{minipage}
%--------------------------------------

%Полосы
%--------------------------------------
\vspace{-25pt}
\hspace{-35pt}\rule{\textwidth}{2.3pt}

\vspace*{-20.3pt}
\hspace{-35pt}\rule{\textwidth}{0.4pt}
%--------------------------------------

\vspace{1.5ex}
\hspace{-35pt} \noindent \small ФАКУЛЬТЕТ\hspace{80pt} <<Информатика и системы управления>>

\vspace*{-16pt}
\hspace{47pt}\rule{0.83\textwidth}{0.4pt}

\vspace{0.5ex}
\hspace{-35pt} \noindent \small КАФЕДРА\hspace{50pt} <<Теоретическая информатика и компьютерные технологии>>

\vspace*{-16pt}
\hspace{30pt}\rule{0.866\textwidth}{0.4pt}
  
\vspace{11em}


\begin{center}
\Large {\bf Лабораторная работа № 2} \\ 
\large {\bf по курсу <<Теория формальных языков>>} \\ 
\end{center}\normalsize

\vspace{8em}

\begin{flushright}
  {Студентка группы ИУ9-52Б Хаустова М. М.\hspace*{15pt} \\
  \vspace{2ex}
  Преподаватель Непейвода А. Н.\hspace*{15pt}}
\end{flushright}

\vfill

\begin{center}
\textsl{Москва 2025}
\end{center}
\end{titlepage}

\section*{Вариант}
Регулярное выражение (вариант):
\[
R = \bigl(aba \mid bab \mid aabb\bigr)^* \; (a\mid b)(a\mid b)\; b b a \; \bigl(aba \mid bab \mid aabb\bigr)^*.
\]
Алфавит: $\Sigma=\{a,b\}$.

\section{Построение НКА}


\subsection{Графическая схема НКА}

\includegraphics[width=\textwidth]{nka.png}

\subsection{Примерная таблица эквивалентностей для НКА}

\includegraphics[width=\textwidth]{nka_table.png}
Ниже приведён фрагмент таблицы различимости, где символ \texttt{+} обозначает подходящие слова, а \texttt{-} - слова, которые не принадлежат языку. Это примерная таблица, из нее можно получить нижнетреугольную путем переставления суффиксов и префиксов между собой.

\section{Построение ДКА}
\subsection{Графическая схема ДКА}


\includegraphics[width=\textwidth]{dka.png}
\subsection{Таблица классов эквивалентности}

Приведён фрагмент таблицы различимости, где символ \texttt{+} обозначает подходящие слова, а \texttt{-} - слова, которые не принадлежат языку.

\begin{center}
\includegraphics[width=\textwidth]{tableDKA}
\end{center}

Анализ полной таблицы различимости (44 состояния) показал, что \emph{все пары различных состояний различимы}.  
Следовательно, ни одно состояние ДКА нельзя объединить с другим без изменения языка.  
Таким образом, исходный ДКА является \textbf{минимальным} по числу состояний.

\section{Построение ПКА}

\includegraphics[width=\textwidth]{pka.png}

В этом графе две ветви - одна НКА, другая - упрощенный НКА. НКА и так минимален, поэтому еще одна таблица эквивалентностей не нужна.


\section{Получение расширенного регулярного выражения}

Исходное регулярное выражение имеет вид:
\[
R = (aba \mid bab \mid aabb)^* (a \mid b)(a \mid b)bba(aba \mid bab \mid aabb)^*.
\]

Данное выражение описывает множество всех слов над алфавитом $\{a, b\}$, которые содержат
подпоследовательность двух произвольных букв, за которой следует фрагмент \texttt{bba},
а также могут иметь произвольное количество блоков \texttt{aba}, \texttt{bab} или \texttt{aabb}
до и после этой части.

Для упрощения и приведения выражения к \emph{расширенному виду} используем доступные операции:
\begin{itemize}
  \item символ подстановки \texttt{.}, соответствующий любому символу алфавита;
  \item маркеры начала и конца строки (\texttt{\^} и \texttt{\$});
  \item необязательную группировку с помощью скобок \texttt{(?:\,)} при необходимости;
  \item квантификаторы итерации (\texttt{*}, \texttt{+}, \texttt{?}).
\end{itemize}

Подвыражение $(a \mid b)(a \mid b)$ эквивалентно записи \texttt{..},
так как каждая точка соответствует ровно одному символу из алфавита $\{a, b\}$.

Таким образом, получаем эквивалентное расширенное регулярное выражение:
\[
R_{\mathrm{ext}} = \; ^ \; (aba \mid bab \mid aabb)^* \; .. \; bba \; (aba \mid bab \mid aabb)^* \; \$.
\]

В записи, удобной для реализации (например, в синтаксисе POSIX или PCRE), оно имеет вид:
\[
\texttt{\^{\char`(}?:aba|bab|aabb{\char`)}*..bba{\char`(}?:aba|bab|aabb{\char`)}*\$}.
\]

\paragraph{Обоснование.}
Расширенное выражение описывает тот же язык, что и исходное, поскольку операция замены
$(a \mid b)(a \mid b)$ на \texttt{..} не изменяет множество допускаемых слов,
а добавление маркеров начала и конца строки гарантирует полное соответствие входной строки языку.



\section{Результаты тестирования}
Скрипт запускается локально: при 5000 случайных тестах длиной до 20 символов на тестовой реализации (локально) несоответствий не зафиксировано — это даёт подтверждение корректности построенных распознавателей.

\end{document}
